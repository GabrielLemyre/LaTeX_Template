% SHORTCUTS DOCUMENT

% Conditionnelle
\def\cond#1#2{\left. #1 \vphantom{#2} \right\vert #2}

%operateur e
\newcommand{\e}[1]{e^{#1}}

% Espérance 
\long\def\Esp#1{\mathds{E}\!\left[ #1 \right]}
\def\esp#1{\Esp{#1}}

% Espérance Conditionnelle et indicée
\long\def\espcond#1#2{\Esp{\cond{#1}{#2}}}

% Espérance Conditionnelle et indicée
\long\def\EspInd#1#2{\mathds{E}_{#1}\!\left[ #2 \right]}

% Espérance Conditionnelle et indicée
\long\def\EspCondInd#1#2#3{\mathds{E}_{#1}\!\left[ \left. \vphantom{#3}#2 \right\vert #3 \right]}


% Variance
\long\def\var#1{\mathds{V}\!ar\!\left[ #1 \right]}

% Variance Conditionnelle
\long\def\varcond#1#2{\mathds{V}\!ar\!\left[ \cond{#1}{#2} \right]}

% Covariance
\long\def\cov#1#2{\hspace{0.08cm}\mathds{C}\!ov\!\left[ #1 ,#2\right]}

% Corrélation
\long\def\cor#1#2{\hspace{0.08cm}corr\!\left(#1,#2\right)}

% POUR TOUT
\def\pourtoutd#1#2{\forall{\, #1 \in{#2} } }

% AUTOCOVARIANCE FUNCTION
\def\avf#1#2{\hspace{0.08cm}\gamma_{#1}\!\left(#2\right)}

% AUTOCOVARIANCE FUNCTION de x
\def\avfx#1{\hspace{0.08cm}\gamma_{x}\!\left(#1\right)}

% AUTOCORRELATION FUNCTION
\def\acf#1#2{\hspace{0.08cm}\rho_{#1}\!\left(#2\right)}

% AUTOCORRELATION FUNCTION de x
\def\acfx#1{\hspace{0.08cm}\rho_{x}\!\left(#1\right)}

%somme avec limite inf et sup et le contenu comme dernier argument
\def\somme#1#2#3{\sum\limits_{#1}^{#2} {#3}}

% EQUATIONS AUX DIFFéRENCES HOMOGÈNES
\def\edh{équation aux Différences Homogènes }

% EQUATIONS AUX DIFFéRENCES HOMOGÈNES
\def\edhs{équations aux Différences Homogènes }

% FONT FOR R FUNCTION
\newcommand{\Rfont}[1]{\texttt{#1}} 

% Nouveau cos
\long\def\coss#1{\cos\!\left( #1 \right)}

% Nouveau sin
\long\def\sins#1{\sin\!\left( #1 \right)}

% Nouveau tan
\long\def\tans#1{\tan\!\left( #1 \right)}

% Nouveau arctan
\long\def\arctans#1{\arctan\!\left( #1 \right)}



\def\rt{$\{r_t\}$\ }
\def\wt{$\{w_t\}$\ }
\def\xtm{$X^{t}_{t+m}$}
\def\sigw{\sigma_{\omega}^2}

\newcommand*{\ovA}[1]{%
  \overline{\mbox{#1}\raisebox{4.5mm}{}}
}

\def\Rconj{\ovA{$R$}}

% PAGE REFERENCING FORMAT
\def\MPN#1{\!$^{(p.#1)}$}

% PROBABILITY
\def\Pr#1{\mathds{P}\!\left[#1\right]}

% MOMENT GENERATING FUNCITON
\def\fgm#1#2{\mathcal{M}_{#1}\!\left(#2\right)}

% Calcul en un point
\def\CalcPt#1#2{\left. #1 \right\vert_{#2}} % EVALUATION

% Derives partielle
\def\partder#1#2{\frac{\partial #2}{\partial #1} } % DERIVE PARTIELLE ORDRE 1 
\def\partderbar#1#2#3{\CalcPt{\frac{\partial}{\partial #1} #2}{#3}} % DERIVE PARTIELLE ORDRE 1 EN UN POINT
\def\tpartder#1#2#3{\frac{\partial^{\left(#1\right)}}{\partial #2^{\left(#1\right)}} #3} % DERIVE PARTIELLE ORDRE T
\def\tpartderbar#1#2#3#4{\CalcPt{\frac{\partial^{\left(#1\right)}}{\partial #2^{\left(#1\right)}} #3}{#4}} % DERIVE PARTIELLE ORDRE T EN UN POINT

%Réels
\def\R{\mathds{R}}
%Naturels
\def\N{\mathds{N}}
%complexe
\def\C{\mathds{C}}
%Rationnels
\def\Q{\mathds{Q}}

\def\parent#1{\left(#1\right)}
\def\crochet#1{\left[#1\right]}

\def\expMoinsUn#1{{#1}^{-1}}
\def\T#1{#1'}

% APPARENCE VECTEUR
\newcommand{\vecc}[1]{\boldsymbol{\mathrm{#1}}}
\newcommand{\vecl}[1]{\vec{#1}}

\newcommand{\matrice}[1]{\begin{bmatrix} #1 \end{bmatrix}}
\newcommand{\vmatrice}[1]{\begin{vmatrix} #1 \end{vmatrix}}

% MATRICE IDENTITE
\def\Idn#1{\vecc{\mathrm{I}}_{#1}}

% Vecteur de UN
\def\vecun#1{\vecc{1}_{#1}}

% Probabilité abréviée
\def\prob#1{p\!\parent{#1}}

% Function syntax
\newcommand{\fun}[3]{#1\!\parent{#2;#3}}

% ______________________________________________________________________________________________
% //////////////////////////////////////////////////////////////////////////////////////////////////////////
% DISTRIBUTIONS PROBABILITÉS -------------------------------------------------------------
% \\\\\\\\\\\\\\\\\\\\\\\\\\\\\\\\\\\\\\\\\\\\\\\\\\\\\\\\\\\\\\\\\\\\\\\\\\\\\\\\\\\\\\\\\\\\\\\\\\\\\\\\\\
% ______________________________________________________________________________________________

% NORMALE
\def\Normale#1#2#3{\mathcal{N}\!\parent{#1_{#3},#2_{#3}}}

% ______________________________________________________________________________________________
% //////////////////////////////////////////////////////////////////////////////////////////////////////////
% STT6615 - SERIES CHRONOLOGIQUES -------------------------------------------------------------
% \\\\\\\\\\\\\\\\\\\\\\\\\\\\\\\\\\\\\\\\\\\\\\\\\\\\\\\\\\\\\\\\\\\\\\\\\\\\\\\\\\\\\\\\\\\\\\\\\\\\\\\\\\
% ______________________________________________________________________________________________


% Modele AR(p)
\def\ar#1{$\mathrm{AR}\!\left(#1\right)$}
% Modele MA(p)
\def\ma#1{$\mathrm{MA}\!\left(#1\right)$}
% Modele ARMA(p,q)
\def\arma#1{$\mathrm{ARMA}\!\left(#1\right)$}
% Modele ARCH(p)
\def\arch#1{$\mathrm{ARCH}\!\left(#1\right)$}
% Modele GARCH(p)
\def\garch#1#2{$\mathrm{GARCH}\!\left(#1,#2\right)$}
% Modele ARIMA(p,d,q)
\def\arima#1{$\mathrm{ARIMA}\!\left(#1\right)$}
% Operateur meilleure prevision linéaire
\def\P#1#2{P_{\! #1}\!\left(X_{#2}\right)}

\def\prev#1#2#3{{#1}_{#2}^{#3}}



% ______________________________________________________________________________________________
% //////////////////////////////////////////////////////////////////////////////////////////////
% STT6415 - REGRESSION ------------------------------------------------------------------------
% \\\\\\\\\\\\\\\\\\\\\\\\\\\\\\\\\\\\\\\\\\\\\\\\\\\\\\\\\\\\\\\\\\\\\\\\\\\\\\\\\\\\\\\\\\\\\\
% ______________________________________________________________________________________________


% Matrix Trace
\def\tr#1{\mathrm{tr}\!\left(#1\right)}

% MSE Function format
\def\MSE#1{\mathrm{MSE}\!\left(#1\right)}

\def\y{\vecc{y}}
\def\yT{\T{\vecc{y}}}
\def\X{\vecc{X}}
\def\B{\vecc{\beta}}
\def\BT{\T{\B}}
\def\Bh{\widehat{\B}}
\def\Bhs{\Bh_\star}
\def\tX{\vecc{\tilde{X}}}
\def\XT{\T{\vecc{X}}}
\def\tXT{\T{\vecc{\tilde{X}}}}
\def\g{\vecc{\gamma}}
\def\gh{\widehat{\vecc{\gamma}}}
\def\muh{\widehat{\vecc{\mu}}}
\def\A{\vecc{\mathrm{A}}}
\def\AT{\T{\A}}
\def\XXT{\expMoinsUn{\left(\XT \X \right)}}



\def\Xu{\X_{\!1}}
\def\Xd{\X_{\!2}}
\def\XTu{\T{\Xu}}
\def\XTd{\T{\Xd}}
\def\XXTu{\expMoinsUn{\left(\XTu \Xu \right)}}
\def\XXTd{\expMoinsUn{\left(\XTu \Xd \right)}}

\def\hu{\vecc{\mathrm{H}}_{1}}
\def\h{\vecc{\mathrm{H}}}
\def\Bo{\B_{0}}
\def\Bu{\B_{1}}
\def\Bd{\B_{2}}

\def\Bru{\B_{1R}}
\def\Brd{\B_{2R}}
\def\Bfu{\B_{1F}}
\def\Bfd{\B_{2F}}

\def\Bruh{\Bh_{1R}}
\def\Brdh{\Bh_{2R}}
\def\Bfuh{\Bh_{1F}}
\def\Bfdh{\Bh_{2F}}

\def\BTu{\T{\B}_{1}}
\def\BTd{\T{\B}_{2}}

\def\BTru{\T{\B}_{1R}}
\def\BTrd{\T{\B}_{2R}}
\def\BTfu{\T{\B}_{1F}}
\def\BTfd{\T{\B}_{2F}}

\def\muzh{\widehat{\mu}^0_R}


\def\BuhOLS{\Bh_{1,OLS}}
\def\BuhWLS{\Bh_{1,WLS}}


% overline sigma
\def\osig{\overline{\sigma^2}}

% DEVOIR 2
\def\yi#1{{y_i}_{#1}}

% ______________________________________________________________________________________________
% //////////////////////////////////////////////////////////////////////////////////////////////
% STT6705 - MODELES DE MARKOV A VARIABLE LATENTE ------------------------------------------------------------------------
% \\\\\\\\\\\\\\\\\\\\\\\\\\\\\\\\\\\\\\\\\\\\\\\\\\\\\\\\\\\\\\\\\\\\\\\\\\\\\\\\\\\\\\\\\\\\\\
% ______________________________________________________________________________________________

\def\vpi{\vecl{\pi}}



% ______________________________________________________________________________________________
% //////////////////////////////////////////////////////////////////////////////////////////////
% ------------------------------ NE PAS MODIFIER CE DOCUMENT ----------------------------------
% \\\\\\\\\\\\\\\\\\\\\\\\\\\\\\\\\\\\\\\\\\\\\\\\\\\\\\\\\\\\\\\\\\\\\\\\\\\\\\\\\\\\\\\\\\\\\\
% ______________________________________________________________________________________________


\usepackage{amsmath}
\usepackage{amsfonts}
\usepackage{amssymb}
\usepackage{amsthm}
\usepackage{amsbsy}
\usepackage{color}
%\usepackage{dsfont}
\usepackage{rotating}

% NEW CALLIGRAPHY FOR MATH SYMBOLS
%\usepackage{calligra}
%\DeclareMathAlphabet{\mathcalligra}{T1}{calligra}{m}{n}

% Permet de cross-out un morceau d'une equation dans textmode et mathmode sauf \cancelto{\infty}{} qui fonctionne juste en mathmode
\usepackage[makeroom]{cancel}


\usepackage{adjustbox} % adjusting table (too wide) 
\usepackage{rotating} % sidewaytables

% Makes the env for landscape
\usepackage{lscape}

\usepackage{calc}

\usepackage{multicol}

% Modification du format des numéros de pages.
\usepackage{fancyhdr}
\usepackage{lastpage}
\usepackage{bold-extra}
%-----------------------------------------------------------------
%-----------------------------------------------------------------
%-----------------------------------------------------------------
\usepackage{trajan}

\newcommand{\CreateurNom}[2]{#1 \scshape{#2}\ }
\newcommand{\CoUn}{\CreateurNom{\PrenomUn}{\NomUn}}
\newcommand{\NomProf}{\CreateurNom{\PrenomProf}{\NomFamilleProf}}

%-----------------------------------------------------------------
% Changement du style de mise en page
	\pagestyle{fancy}
	\fancyhf{}
 	\fancyhead[L]{\sigleCours \ - \TypeRemise}
 	\fancyhead[C]{--- \CoUn  ---}
 	\fancyhead[R]{\dateremise}
 	
 %-----------------------------------------------------------------
 % Modifier style du numéro de page
\rfoot{\vspace*{-2em} \thepage /\pageref{LastPage}}

% TITLE PAGE FORMATING
\newcommand{\TitlePAGE}{
\begin{titlepage}
	
	\centering % Centre everything on the title page
	
	%\scshape % Use small caps for all text on the title page
	
	\vspace*{\baselineskip} % White space at the top of the page
	%\includegraphics[width=0.15\textwidth]{example-image-1x1}\par\vspace{1cm}
	{\huge \scshape{\Establishment} \par}
	
	{\scshape\Large \Department \par}
	
	\vspace{2cm}
	
	\rule{\textwidth}{1.6pt}\vspace*{-\baselineskip}\vspace*{2pt} % Thick horizontal rule
	\rule{\textwidth}{0.4pt} % Thin horizontal rule
	
	\vspace{0.75\baselineskip} % Whitespace above the title
	
	{\LARGE \sigleCours \\ \scshape{\titreCours} \par}
	\rule{\widthof{\TypeRemise}+\widthof{\TypeRemise}}{1pt} \\
	{\scshape\Large \TypeRemise \par}
	
	\vspace{0.75\baselineskip} % Whitespace below the title
	
	\rule{\textwidth}{0.4pt}\vspace*{-\baselineskip}\vspace{3.2pt} % Thin horizontal rule
	\rule{\textwidth}{1.6pt} % Thick horizontal rule
	
	\vspace{2cm}
	{\Large \CoUn \par}
	\vfill
	Remis à \par
	{\Large \NomProf \\}

	\vfill

% Bottom of the page
	{\large \dateremise \par}
\end{titlepage}
\clearpage
}
%-----------------------------------------------------------------
% Permet d'ajouter entre autre, une ligne à gauche des exemples,théorèmes,etc...


\usepackage[framemethod=TikZ]{mdframed}
\newcounter{Question}
\newcounter{subQcounter}[Question]

\newenvironment{Question}[1][]{%
	\refstepcounter{Question}% increment the environment's counter
	\vspace{60pt}
	\begin{mdframed}[%
		frametitle={{\huge{#1}}},
		%frametitlebackgroundcolor=gray!30,
	    topline=true,
	    frametitlerule=true,
	    bottomline=true,
	    rightline=false,
	    leftline=true,
	    nobreak=false,
	    innerbottommargin=1em,
		%frametitleaboveskip=10pt, frametitlebelowskip=10pt,
		font=\Large
	]%
\quad }{%
    \end{mdframed}
}

% FORMAT DU COMPTEUR DE SOUS QUESTION
\renewcommand\thesubQcounter{\alph{subQcounter}}

\newenvironment{SousQuestion}[1][]{%
	\refstepcounter{subQcounter}
	~\\
	~\\
	\fbox{
	\begin{minipage}{1.0\textwidth}
	\textbf{$\left(\mathrm{\textbf{\thesubQcounter}}\right)$ #1}
	\end{minipage}}\\
	\par
	}{
}

\newenvironment{Cas}[1][]{%
	\begin{mdframed}[%
		frametitle={#1},
		frametitlebackgroundcolor=gray!30,
	    topline=true,
	    frametitlerule=true,
	    bottomline=true,
	    rightline=true,
	    leftline=true,
	    nobreak=false,
	    %innerleftmargin=1.3em,
		%frametitleaboveskip=10pt, frametitlebelowskip=10pt,
		font=\Large
	]%
\quad }{%
    \end{mdframed}
}

%-----------------------------------------------------------------
%------------------------------
%Environnements
%------------------------------

%Format des théorèmes/definitions et exemples
\theoremstyle{definition}

%--Format de nouvel environnement
\usepackage{thmtools}
%\newtheoremstyle{<name>}%
 % spaceabove={<space above>}%
 % spacebelow={<space below>}%
 % {<body font>}%
 % {<indent amount>}%
 % headfont={<theorem head font>}%
 % {<punctuation after theorem head>}%
 % {<space after theorem>}%

%Format pour des adresse web
\usepackage{url}

%
\usepackage{upgreek}

%Tableaux et figures
\usepackage{tabularx}
\usepackage{graphicx}
\usepackage{caption}
\usepackage{booktabs}
\usepackage{wrapfig}

%Placer des ancres dans le texte et y faire référence
\usepackage[pageanchor]{hyperref}
%Permet que les numéros de pages de la ToC soit des hyperliens
%\usepackage[linktocpage=true]{hyperref}
\usepackage{tocloft}
\renewcommand{\cftparapresnum}{\S}



%Pour prendre du code R et l'inclure en format LaTeX
\usepackage{listings}

%-----------------------------------------------------------------
\definecolor{dkgreen}{rgb}{0,0.4,0}
\definecolor{gray}{rgb}{0.5,0.5,0.5}
\definecolor{mauve}{rgb}{0.58,0,0.82}

% Inclure code R dans le texte -------------------------
\lstset{ %
  language=R,                     % the language of the code
  basicstyle=\footnotesize,       % the size of the fonts that are used for the code
  numbers=left,                   % where to put the line-numbers
  numberstyle=\tiny\color{gray},  % the style that is used for the line-numbers
  stepnumber=1,                   % the step between two line-numbers. If it's 1, each line
                                  % will be numbered
  numbersep=5pt,                  % how far the line-numbers are from the code
  backgroundcolor=\color{white},  % choose the background color. You must add \usepackage{color}
  showspaces=false,               % show spaces adding particular underscores
  showstringspaces=false,         % underline spaces within strings
  showtabs=false,                 % show tabs within strings adding particular underscores
  frame=single,                   % adds a frame around the code
  rulecolor=\color{black},        % if not set, the frame-color may be changed on line-breaks within not-black text (e.g. commens (green here))
  tabsize=2,                      % sets default tabsize to 2 spaces
  captionpos=b,                   % sets the caption-position to bottom
  breaklines=true,                % sets automatic line breaking
  breakatwhitespace=false,        % sets if automatic breaks should only happen at whitespace
  title=\lstname,   				 % show the filename of files included with \lstinputlisting;
  %basicstyle=\large,
  					                 % also try caption instead of title
  keywordstyle=\color{blue},      % keyword style
  commentstyle=\color{dkgreen},   % comment style
  stringstyle=\color{dkgreen},      % string literal style
  escapeinside={\%*}{*)},         % if you want to add a comment within your code
  morekeywords={*,...}            % if you want to add more keywords to the set
} 
%------------------------------
%espacement
%------------------------------
\usepackage{setspace}
%Puis changer les option d'espacement:
%\doublespacing
%\singlespacing
%\onehalfspacing
%\setstretch{1.2}

%\newcolumntype{C}{ >{\centering\arraybackslash} m{2cm} }
%\newcolumntype{d}{ >{\centering\arraybackslash} m{0.7cm} }
\usepackage{xfrac}

\usepackage{colortbl}
\definecolor{silver}{RGB}{220,220,220}


